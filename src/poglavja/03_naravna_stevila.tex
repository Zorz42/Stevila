\begin{definicija}
    Peanovi Aksiomi:
    \begin{itemize}
        \item $0 \in \mathbb{N}$
        \item obstaja preslikava: $s: \mathbb{N} \to \mathbb{N}$, ki je injektivna
        \item $0 \notin s(\mathbb{N})$
        \item $(A \subseteq \mathbb{N} \wedge 0 \in A \wedge n \in A \implies s(n) \in A) \implies A = \mathbb{N}$
    \end{itemize}
\end{definicija}

\begin{trditev}
    Obstaja natanko ena množica, ki zadošča Peanovim Aksiomom.
\end{trditev}
\begin{dokaz}
Vsaj ena množica obstaja zaradi aksioma neskončnosti in aksioma izbire.
Predpostavimo, da obstajata dve množici, ki zadoščata Peanovim Aksiomom: $\mathbb{N}_1$ in $\mathbb{N}_2$.
V zadnji aksiom vstavimo $\mathbb{N}_1$ in $\mathbb{N}_2$ in dobimo $\mathbb{N}_1 = \mathbb{N}_2$.
\end{dokaz}
