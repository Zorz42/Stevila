% Definicija relacije
\begin{definicija}
    Relacija $R$ je množica urejenih parov, tj. $R \subseteq A \times B$.
    Namesto $(a, b) \in R$ pišemo $aRb$.
\end{definicija}

% Definicija preslikave
\begin{definicija}
    Relacija $f \in A \times B$ je preslikava, če $\forall x \in X(\exists! y \in Y(xfy))$.
    Namesto $xfy$ pišemo $f(x) = y$, ker ima $f(x)$ natanko eno vrednost za vsak $x$.
    Namesto $f \in A \times B$ pišemo $f: A \to B$.
\end{definicija}

% Definicija domene in slike preslikave
\begin{definicija}
    Naj bo $f: A \to B$ preslikava. Množica $A$ se imenuje domena preslikave $f$ in se označuje z $\text{dom}(f)$.
    Množica $\{f(x) | x \in A\}$ se imenuje slika preslikave $f$ in se označuje z $\text{im}(f)$.
    Očitno velja $\text{im}(f) \subseteq B$.
\end{definicija}

% Definicija injektivne, surjektivne in bijektivne preslikave
\begin{definicija}
    \begin{enumerate}
        \item Preslikava $f: A \to B$ je injektivna, če $\forall x, y \in A(f(x) = f(y) \implies x = y)$.
        \item Preslikava $f: A \to B$ je surjektivna, če $\forall y \in B(\exists x \in A(f(x) = y))$.
        Iz tega direktno sledi: $f$ je surjektivna $\iff \text{im}(f) = B$.
        \item Preslikava $f: A \to B$ je bijektivna, če je injektivna in surjektivna.
    \end{enumerate}
\end{definicija}

% Definicija kompozituma preslikav
\begin{definicija}
    Naj bosta $f: A \to B$ in $g: B \to C$ preslikavi.
    Kompozitum preslikav $f$ in $g$ je preslikava $g \circ f: A \to C$, definirana s predpisom $(g \circ f)(x) = g(f(x))$.
\end{definicija}

% Definicija identične preslikave
\begin{definicija}
    Naj bo $A$ množica.
    Identiteta na množici $A$ je preslikava $\text{id}_A: A \to A$, definirana s predpisom $\text{id}_A(x) = x$.
\end{definicija}


\begin{trditev}
    Naj bodo $f: A \to B$, $g: B \to C$ in $h: C \to D$ preslikave.
    \begin{enumerate}
        \item Kompozitum preslikav je asociativen: $(h \circ g) \circ f = h \circ (g \circ f)$.
        \item Identiteta je enota za kompozitum: $\text{id}_B \circ f = f = f \circ \text{id}_A$.
    \end{enumerate}
\end{trditev}
\begin{dokaz}
\begin{enumerate}
    \item $(h \circ g) \circ f = h \circ (g \circ f)$
    \begin{align*}
        ((h \circ g) \circ f)(x) &= (h \circ g)(f(x)) = h(g(f(x))) \\
        (h \circ (g \circ f))(x) &= h((g \circ f)(x)) = h(g(f(x)))
    \end{align*}
    \item $\text{id}_B \circ f = f = f \circ \text{id}_A$
    \begin{align*}
        (\text{id}_B \circ f)(x) &= \text{id}_B(f(x)) = f(x) \\
        (f \circ \text{id}_A)(x) &= f(\text{id}_A(x)) = f(x)
    \end{align*}
\end{enumerate}
\end{dokaz}

% Definicija levega in desnega inverza
\begin{definicija}
    Naj bo $f: A \to B$ preslikava.
    Preslikava $g: B \to A$ je levi inverz preslikave $f$, če je $g \circ f = \text{id}_A$.
    Preslikava $h: B \to A$ je desni inverz preslikave $f$, če je $f \circ h = \text{id}_B$.
    Če je inverz levi in desni, mu pravimo samo inverz in ga označimo z $f^{-1}$.
\end{definicija}

\begin{trditev}
    Naj bo $f$ preslikava.
    $f$ ima levi inverz $\iff f$ je injektivna.
\end{trditev}
\begin{dokaz}
$\implies$: \\
Naj bo $g$ levi inverz preslikave $f$.
$f(x) = f(y) \implies g(f(x)) = g(f(y)) \implies x = y$. \\

$\impliedby$: \\
Naj bo $f$ injektivna.
Definirajmo relacijo $g \subseteq \text{im}(f) \times A$ s predpisom $g = \{(f(x), x) | x \in A\}$.
Dokažimo, da je $g$ funkcija:
\begin{itemize}
    \item $\forall y \in \text{im}(f) \exists x \in A(y = f(x))$, ker je $y \in \text{im}(f)$.
    \item $\forall y \in \text{im}(f) \exists! x \in A(y = f(x))$, ker je $f$ injektivna.
\end{itemize}
Dokažimo, da je $g$ levi inverz preslikave $f$:
\begin{align*}
    (g \circ f)(x) &= g(f(x)) = g(y) = x
\end{align*}
\end{dokaz}

\begin{trditev}
    Naj bo $f$ preslikava.
    $f$ ima desni inverz $\iff f$ je surjektivna.
\end{trditev}
\begin{dokaz}
$\implies$: \\
Naj bo $g$ desni inverz preslikave $f$.
$f(g(x)) = x$ iz tega sledi, da za vsak $y \in B$ obstaja $x \in A$, da je $f(x) = y$, saj je to kar $g(y)$. \\

$\impliedby$: \\
Naj bo $f$ surjektivna.
Definirajmo relacijo $g \subseteq B \times A$ s predpisom $g = \{(f(x), x) | x \in A\}$.
Dokažimo, da je $g$ funkcija:
\begin{itemize}
    \item $\forall y \in B \exists! x \in A(y = f(x))$, ker je $f$ surjektivna funkcija.
\end{itemize}
\end{dokaz}

\begin{trditev}
    Če obstajata levi inverz in desni inverz preslikave $f$, sta enaka.
\end{trditev}
\begin{dokaz}
\begin{align*}
    g \circ f &= \text{id}_A \\
    f \circ h &= \text{id}_B \\
    g = g \circ \text{id}_B &= g \circ (f \circ h) = (g \circ f) \circ h = \text{id}_A \circ h = h
\end{align*}
\end{dokaz}

\begin{trditev}
    $f$ je bijektivna preslikava $\iff f$ ima inverz.
\end{trditev}
\begin{dokaz}
\begin{align*}
    f \text{ je bijektivna} &\iff f \text{ je injektivna in surjektivna} \\
    &\iff f \text{ ima levi inverz in desni inverz} \\
    &\iff f \text{ ima inverz}
\end{align*}
\end{dokaz}

\begin{trditev}
    Če ima $f$ inverz, je en sam.
    Označimo ga z $f^{-1}$.
\end{trditev}
\begin{dokaz}
Naj bosta $g$ in $h$ inverza preslikave $f$.
Ker je $g$ levi inverz, in $h$ desni inverz, je $g = h$.
\end{dokaz}

\begin{definicija}
    Naj bo $f: A \to B$ preslikava.
    Naj bo $A' \subseteq A$.
    $f(A') = \{f(x) | x \in A'\}$.
\end{definicija}

