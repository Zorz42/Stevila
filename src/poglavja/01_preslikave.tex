% Definicija relacije
\begin{definicija}
    Relacija $R$ je množica urejenih parov, tj. $R \subseteq A \times B$.
    Namesto $(a, b) \in R$ pišemo $aRb$.
\end{definicija}

% Definicija preslikave
\begin{definicija}
    Relacija $f \in A \times B$ je preslikava, če $\forall x \in X(\exists! y \in Y(xfy))$.
    Namesto $xfy$ pišemo $f(x) = y$, ker ima $f(x)$ natanko eno vrednost za vsak $x$.
    Namesto $f \in A \times B$ pišemo $f: A \to B$.
\end{definicija}

% Definicija domene in slike preslikave
\begin{definicija}
    Naj bo $f: A \to B$ preslikava. Množica $A$ se imenuje domena preslikave $f$ in se označuje z $\text{dom}(f)$.
    Množica $\{f(x) | x \in A\}$ se imenuje slika preslikave $f$ in se označuje z $\text{im}(f)$.
    Očitno velja $\text{im}(f) \subseteq B$.
\end{definicija}

% Definicija injektivne, surjektivne in bijektivne preslikave
\begin{definicija}
    \begin{enumerate}
        \item Preslikava $f: A \to B$ je injektivna, če $\forall x, y \in A(f(x) = f(y) \implies x = y)$.
        \item Preslikava $f: A \to B$ je surjektivna, če $\forall y \in B(\exists x \in A(f(x) = y))$.
        Iz tega direktno sledi: $f$ je surjektivna $\iff \text{im}(f) = B$.
        \item Preslikava $f: A \to B$ je bijektivna, če je injektivna in surjektivna.
    \end{enumerate}
\end{definicija}

% Definicija kompozituma preslikav
\begin{definicija}
    Naj bosta $f: A \to B$ in $g: B \to C$ preslikavi. Kompozitum preslikav $f$ in $g$ je preslikava $g \circ f: A \to C$, definirana s predpisom $(g \circ f)(x) = g(f(x))$.
\end{definicija}

% Definicija identične preslikave
\begin{definicija}
    Naj bo $A$ množica. Identiteta na množici $A$ je preslikava $\text{id}_A: A \to A$, definirana s predpisom $\text{id}_A(x) = x$.
\end{definicija}

% Definicija levega in desnega inverza
\begin{definicija}
    Naj bo $f: A \to B$ preslikava.
    Preslikava $g: B \to A$ je levi inverz preslikave $f$, če je $g \circ f = \text{id}_A$.
    Preslikava $h: B \to A$ je desni inverz preslikave $f$, če je $f \circ h = \text{id}_B$.
    Če je inverz levi in desni, mu pravimo samo inverz in ga označimo z $f^{-1}$.
\end{definicija}

\begin{trditev}
    Naj bo $f$ preslikava.
    $f$ ima levi inverz $\iff f$ je injektivna.
\end{trditev}
Dokaz: \\
$\implies$: \\
Naj bo $g$ levi inverz preslikave $f$.
$f(x) = f(y) \implies g(f(x)) = g(f(y)) \implies x = y$. \\

