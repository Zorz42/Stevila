\begin{definicija}
    Definiramo relacijo $\sim$ na množici $\mathbb{N} \times \mathbb{N}$. \\
    Definiramo: $(a, b) \sim (c, d) \iff a + d = b + c$
\end{definicija}

\begin{trditev}
    $\sim$ je ekvivalenčna relacija.
\end{trditev}
\begin{dokaz}
    \begin{itemize}
        \item $a \sim a$ \\
        $(a, b) \sim (a, b) \iff a + b = b + a$
        \item $a \sim b \iff b \sim a$ \\
        $(a, b) \sim (c, d) \iff a + d = b + c \iff c + b = a + d \iff (c, d) \sim (a, b)$
        \item $a \sim b \wedge b \sim c \implies a \sim c$ \\
        $(a_1, a_2) \sim (b_1, b_2) \wedge (b_1, b_2) \sim (c_1, c_2) \implies$ \\
        $a_1 + b_2 = b_1 + a_2 \wedge b_1 + c_2 = b_2 + c_1 \implies$ \\
        $a_1 + b_2 + b_1 + c_2 = b_1 + a_2 + b_2 + c_1 \implies$ \\
        $a_1 + c_2 = a_2 + c_1 \implies (a_1, a_2) \sim (c_1, c_2)$
    \end{itemize}
\end{dokaz}

\begin{definicija}
    Z $\mathbb{Z}$ označimo vse ekvivalenčne razrede $\mathbb{N} \times \mathbb{N}$ pod relacijo $\sim$.
\end{definicija}

\begin{trditev}
    Za vsak $(a, b)$ obstaja nek $c$, da velja $(0, c) \sim (a, b)$ ali pa $(c, 0) \sim (a, b)$.
\end{trditev}
\begin{dokaz}
    Dokaz z indukcijo na $a$.
    1. $a = 0$ dobimo $(0, b) \sim (0, c)$ za $b = c$ \\
    2. $a \to a + 1$.
    Obravnavamo po primerih: \\
    2.1. $(a, b) \sim (0, c)$ in $c \neq 0$ \\
    $a + c = b \implies a + c' + 1 = b \implies (a + 1, b) \sim (0, c')$ \\
    2.2. $(a, b) \sim (0, 0)$ \\
    $a = b \implies a + 1 = b + 1 \implies (a + 1, b) \sim (1, 0)$ \\
    2.3. $(a, b) \sim (c, 0)$ in $c \neq 0$ \\
    $a = b + c \implies a + 1 = b + c + 1 \implies (a + 1, b) \sim (c + 1, 0)$
\end{dokaz}

\begin{definicija}
    Iz prejšnje trditve vemo, da vsak razred iz $\mathbb{Z}$ vsebuje element oblike $(c, 0)$ ali pa $(0, c)$, zato definiramo: \\
    $a = (a, 0)$ in $-(a, b) = (b, a)$ in tako označimo elemente celih števil.
    Nekatera števila imajo isto oznako kot naravna števila.
    To je zato, ker se enako obnašajo.
    Namesto $\sim$ tudi pišemo $=$, kadar uporabljamo to notacijo.
\end{definicija}

\begin{trditev}
    Vsa števila, ki so ekvivalentna $(0, c)$ so oblike $(d, c + d)$ in podobno za $(c, 0)$
\end{trditev}
\begin{dokaz}
    $(a, b) \sim (0, c) \implies a + c = b \implies (a, b) = (a, c + a)$ \\
    Podoben dokaz za $(c, 0)$.
\end{dokaz}

\begin{definicija}
    Definiramo seštevanje na celih številih po komponentah.
    $(a, b) + (c, d) \sim (a + c, b + d)$
\end{definicija}

\begin{trditev}
    Seštevanje na celih številih je dobro definirano.
\end{trditev}
\begin{dokaz}
    Dokazat je treba, da se vsi elementi dveh ekvivalenčnih razredov seštejejo v elemente istega ekvivalenčnega razreda. \\
    1. $(a, c + a) + (b, c' + b) \sim (a + b, c + c' + a + b)$ je pa očitno v istem razredu. \\
    2. $(a, c + a) + (c' + b, b) \sim (c + a + b, c' + a + b)$ je pa očitno v istem razredu. \\
    Podoben razmislek za 3 in 4
\end{dokaz}

\begin{trditev}
    Invertiranje je dobro definirano.
\end{trditev}
\begin{dokaz}
    Treba je dokazat: $(a, b) \sim (c, d) \iff -(a, b) \sim -(c, d)$: \\
    $(a, b) \sim (c, d) \iff a + d = b + c \iff b + c = a + d \iff (b, a) \sim (d, c) \iff -(a, b) \sim -(c, d)$
\end{dokaz}

\begin{definicija}
    Odštevanje definiramo kot: $a - b = a + (-b)$
\end{definicija}

\begin{trditev}
    $a - a = 0$
\end{trditev}
\begin{dokaz}
    $a - a = (a, 0) + (0, a) \sim (a, a) \sim (0, 0) = 0$
\end{dokaz}

\begin{trditev}
    $(\mathbb{Z}, +)$ tvori abelovo grupo.
\end{trditev}
\begin{dokaz}
    Da tvori komutativen monoid, je trivialno res, saj se ta lastnost podeduje od naravnih števil, saj seštevamo po komponentah. \\
    Obstoj inverza pa dokazuje prejšnja trditev.
\end{dokaz}

\begin{definicija}
    Množenje definiramo na celih številih na naslednji način: \\
    $(a, b) \cdot (c, d) \sim (ac + bd, ad + bc)$ \\
\end{definicija}

\begin{trditev}
    $a(b + c) = ab + ac$
\end{trditev}
\begin{dokaz}
    $a(b + c) = (a_1, a_2)\cdot(b_1 + c_1, b_2 + c_2) = (a_1(b_1 + c_1) + a_2(b_2 + c_2), a_2(b_1 + c_1) + a_1(b_2 + c_2)) =$
    $(a_1 b_1 + a_1 c_1 + a_2 b_2 + a_2 c_2, a_2 b_1 + a_2 c_1 + a_2 b_2 + a_2 c_2) = $
    $(a_1 b_1 + a_2 b_2, a_1 b_2 + a_2 b_1) + (a_1 c_1 + a_2 c_2, a_1 c_2 + a_2 c_1) =$
    $(a_1, a_2)\cdot(b_1, b_2) + (a_1, a_2)\cdot(c_1, c_2) =$
    $ab + ac$
\end{dokaz}

\begin{trditev}
    $(\mathbb{Z}, \cdot)$ tvori komutativni monoid.
\end{trditev}
\begin{dokaz}
    \begin{itemize}
        \item $(ab)c = a(bc)$\\
        $((a_1,a_2)(b_1,b_2))(c_1,c_2) = (a_1 b_1 + a_2 b_2, a_1 b_2 + a_2 b_1)(c_1, c_2) = $
        $((a_1 b_1 + a_2 b_2)c_1 + (a_1 b_2 + a_2 b_1)c_2, (a_1 b_1 + a_2 b_2)c_2 + (a_1 b_2 + a_2 b_1)c_1) = $
        $(a_1 b_1 c_1 + a_2 b_2 c_1 + a_1 b_2 c_2 + a_2 b_1 c_2, a_1 b_1 c_2 + a_2 b_2 c_2 + a_1 b_2 c_1 + a_2 b_1 c_1)$ \\
        Po drugi strani pa:\\
        $(a_1,a_2)((b_1,b_2)(c_1,c_2)) = (a_1,a_2)(b_1 c_1 + b_2 c_2, b_1 c_2 + b_2 c_1) =$
        $(a_1(b_1 c_1 + b_2 c_2) + a_2(b_1 c_2 + b_2 c_1), a_1(b_1 c_2 + b_2 c_1) + a_2(b_1 c_1 + b_2 c_2)) =$
        $(a_1 b_1 c_1 + a_1 b_2 c_2 + a_2 b_1 c_2 + a_2 b_2 c_1, a_1 b_1 c_2 + a_1 b_2 c_1 + a_2 b_1 c_1 + a_2 b_2 c_2)$
        \item $ab = ba$\\
        $(a_1, a_2)(b_1, b_2) = (a_1 b_1 + a_2 b_2, a_1 b_2 + a_2 b_1) = (b_1 a_1 + b_2 a_2, b_1 a_2 + b_2 a_1) = (b_1, b_2)(a_1, a_2)$
        \item $1a = a$\\
        $(a_1, a_2)(1, 0) = (1 a_1 + 0 a_2, 0 a_1 + 1 a_2) = (a_1, a_2)$
    \end{itemize}
\end{dokaz}

\begin{definicija}
    $(a_1, a_2) \leq (b_1, b_2)$
\end{definicija}