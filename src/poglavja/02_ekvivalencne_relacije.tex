\begin{definicija}
    Naj bo $\sim \subseteq A \times A$. $\sim$ je ekvivalenčna relacija, če velja:
    \begin{itemize}
        \item $a\sim a$ (refleksivnost)
        \item $a\sim b \implies b\sim a$ (simetričnost)
        \item $a\sim b \wedge b\sim c \implies a\sim c$ (tranzitivnost)
    \end{itemize}
\end{definicija}

\begin{definicija}
    Naj bo $\sim$ ekvivalenčna relacija na množici $A$.
    Množico $[a] = \{x \in A | x \sim a\}$ imenujemo ekvivalenčni razred elementa $a$.
\end{definicija}

\begin{definicija}
    Naj bo $\sim$ ekvivalenčna relacija na množici $A$.
    $\{[x] | x \in A\}$ označimo z $A/\sim$.
\end{definicija}

\begin{trditev}
    $a \in [a]$
\end{trditev}
Dokaz:
\begin{align*}
    a\sim a &\implies a \in [a]
\end{align*}

\begin{trditev}
    $[a] = [b] \iff a\sim b$
\end{trditev}
Dokaz:\\
$\implies$:
\begin{align*}
    [a] = [b] &\implies \{x \in A | x \sim a\} = \{x \in A | x \sim b\} \\
    &\implies \forall x \in A(x \sim a \iff x \sim b) \\
    &\implies a \sim a \iff a \sim b \\
    &\implies a\sim b
\end{align*}

$\impliedby$:
\begin{align*}
    a\sim b &\implies a \in \{x \in A | x \sim b\} \wedge b \in \{x \in A | x \sim a\}
\end{align*}
Ker $x \sim a \wedge a \sim b \implies x \sim b$, velja: $\{x \in A | x \sim a\} \subseteq \{x \in A | x \sim b\}$.
Podobno velja: $\{x \in A | x \sim b\} \subseteq \{x \in A | x \sim a\}$.
\begin{align*}
    a \in \{x \in A | x \sim b\} \wedge b \in \{x \in A | x \sim a\} &\implies [a] \subseteq [b] \wedge [b] \subseteq [a] \\
    &\implies [a] = [b]
\end{align*}

\begin{trditev}
    $[a] \cap [b] \neq \emptyset \iff [a] = [b]$
\end{trditev}
Dokaz: \\
$\impliedby$: \\
Je očitno, ker noben ekvivalenčen razred ni prazen, saj $[a]$ vsebuje vsaj $a$.\\
\\
$\implies$:\\
Naj bo $[a] \cap [b] \neq \emptyset$.
Vzemimo $x \in [a] \cap [b]$.
Če nista enaka, potem obstaja $y \in [a] \wedge y \notin [b]$.
Če $y \notin [a] \wedge y \in [b]$ velja podoben dokaz, zato bomo gledali samo prvo.

\begin{trditev}
    $A = \bigcup_{x \in A} [x]$
\end{trditev}
Dokaz:
$x \in [x]$, zato je $x \in \bigcup_{x \in A} [x]$.



