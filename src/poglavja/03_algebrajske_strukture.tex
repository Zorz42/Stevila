\begin{definicija}
    Naj bo $A$ množica in $\circ: A \times A \to A$ funkcija. \\
    Namesto $\circ((a, b))$ pišemo $a \circ b$. \\
    Če velja:
    \begin{itemize}
        \item $a \circ (b \circ c) = (a \circ b) \circ c$ (asociativnost)
        \item $\exists e \in A. a \circ e = e \circ a = a$ (obstoj enote)
    \end{itemize}
    potem je $(A, \circ)$ monoid.
\end{definicija}

\begin{definicija}
    Monoid $(A, \circ)$ je komutativen, če velja:
    \begin{itemize}
        \item $a \circ b = b \circ a$
    \end{itemize}
\end{definicija}

\begin{definicija}
    Naj bo $(A, \circ)$ monoid.
    Če dodatno velja:
    \begin{itemize}
        \item $\forall a \in A. \exists a^{-1} \in A. a \circ a^{-1} = a^{-1} \circ a = e$
    \end{itemize}
    potem je $(A, \circ)$ grupa.
\end{definicija}

\begin{definicija}
    Naj bo $(A, \circ)$ grupa.
    Če velja:
    \begin{itemize}
        \item $\forall a, b \in A. a \circ b = b \circ a$
    \end{itemize}
    potem je $(A, \circ)$ Abelova grupa.
\end{definicija}

\begin{trditev}
    Naj bo $(A, \circ)$ monoid.
    Enota je enolično določena.
\end{trditev}
\begin{dokaz}
    Predpostavimo, da obstajata dve enoti: $e$ in $e'$.
    Potem $e = e \circ e' = e'$.
\end{dokaz}

\begin{trditev}
    Naj bo $(G, \circ)$ grupa.
    Potem ima enačba $a \circ x = b$ enolično rešitev za vsak $a, b \in G$.
\end{trditev}
\begin{dokaz}
    \begin{align*}
        a \circ x = b \iff \\
        a^{-1} \circ (a \circ x) = a^{-1} \circ b \iff \\
        (a^{-1} \circ a) \circ x = a^{-1} \circ b \iff \\
        e \circ x = a^{-1} \circ b \iff \\
        x = a^{-1} \circ b
    \end{align*}
\end{dokaz}

\begin{trditev}
    Naj bo $(G, \circ)$ grupa.
    Potem velja: $a\circ b = a\circ c \implies b = c$
\end{trditev}
\begin{dokaz}
    \begin{align*}
        a\circ b = a\circ c \implies \\
        a^{-1} \circ (a \circ b) = a^{-1} \circ (a \circ c) \implies \\
        (a^{-1} \circ a) \circ b = (a^{-1} \circ a) \circ c \implies \\
        e \circ b = e \circ c \implies \\
        b = c
    \end{align*}
\end{dokaz}

\begin{trditev}
    Naj bo $(G, \circ)$ grupa, potem je inverz enolično določen za vsak $a \in G$ posebej.
\end{trditev}
\begin{dokaz}
    Predpostavimo, da obstajata dva inverza: $b$ in $b'$.
    Potem $b = b \circ e = b \circ (a \circ b') = (b \circ a) \circ b' = e \circ b' = b'$
\end{dokaz}

\begin{trditev}
    Naj bo $(G, \circ)$ grupa.
    Potem velja: $e^{-1} = e$.
\end{trditev}
\begin{dokaz}
    $e = e \circ e$, zato je $e$ inverz za $e$.
    Ker je inverz enolično določen, je $e^{-1} = e$.
\end{dokaz}

\begin{trditev}
    Naj bo $(G, \circ)$ grupa.
    Potem za vsak $a \in G$ velja: $(a^{-1})^{-1} = a$.
\end{trditev}
\begin{dokaz}
    \begin{align*}
        (a^{-1})^{-1} \circ a^{-1} = e = a^{-1} \circ a \implies \\
        (a^{-1})^{-1} \circ a^{-1} = a \circ a^{-1} \implies \\
        (a^{-1})^{-1} = a
    \end{align*}
\end{dokaz}

\begin{definicija}
    Naj bo $A$ množica in $\leq \subseteq A \times A$ relacija.
    Potem je $(A, \leq)$ delno urejena množica, če velja:
    \begin{itemize}
        \item $a \leq a$ (refleksivnost)
        \item $a \leq b \wedge b \leq a \implies a = b$ (antisimetričnost)
        \item $a \leq b \wedge b \leq c \implies a \leq c$ (tranzitivnost)
    \end{itemize}
    Če še dodatno velja:
    \begin{itemize}
        \item $a \leq b \vee b \leq a$ (sovisnost)
    \end{itemize}
    je $(A, \leq)$ linearno urejena množica.
\end{definicija}

\begin{definicija}
    $(A, +, \cdot)$ je polje, če sta $(A, +)$ in $(A, \cdot)$ Abelovi grupi, ter velja $a(b + c) = ab + ac$ z izjemo, da $0$ nima inverza.
    Enoto v $+$ označimo z $0$, enoto v $\cdot$ pa označimo z $1$.
\end{definicija}

\begin{trditev}
    Naj bo $(A, +, \cdot)$ polje.
    Potem velja: $a \cdot 0 = 0$
\end{trditev}
\begin{dokaz}
    $a0 = a(0 + 0) = a0 + a0 \implies a0 = 0$
\end{dokaz}

\begin{trditev}
    Naj bo $(A, +, \cdot)$ polje.
    Potem velja: $-1a = -a$
\end{trditev}
\begin{dokaz}
    $0 = a(-1 + 1) = -1a + a \implies -1a = -a$
\end{dokaz}

\begin{trditev}
    Naj bo $(A, +, \cdot)$ polje.
    Potem velja: $(-a)b = a(-b) = -ab$
\end{trditev}
\begin{dokaz}
    $(-a)b = -1ab = -ab = -1ab = a(-1)b = a(-b)$
\end{dokaz}

\begin{definicija}
    $(A, +, \cdot, P)$ je urejena polje, če je $(A, +, \cdot)$ polje, ter velja:
    \begin{itemize}
        \item $a \in P \vee -a \in P$ za vse $a \neq 0$
        \item $0 \notin P$
        \item $a, b \in P \implies a + b \in P$
        \item $a, b \in P \implies ab \in P$
    \end{itemize}
    Če je $a \in P$ mu rečemo pozitivno število.
\end{definicija}

\begin{trditev}
    Ali je $a$ pozitivno ali pa $-a$, ne pa obe za $a \neq 0$.
\end{trditev}
\begin{dokaz}
    Predpostavimo, da je $a$ pozitivno in $-a$ pozitivno.
    Potem je tudi $a + (-a) = 0$ pozitivno, kar je protislovje.
\end{dokaz}

\begin{trditev}
    $1$ je pozitiven element urejenega polja.
\end{trditev}
\begin{dokaz}
    Naj bo $-1$ pozitiven.
    Potem je $(-1)(-1) = 1$ pozitiven, kar je protislovje.
\end{dokaz}

\begin{definicija}
    $a > b \iff a - b \in P$ za neko urejeno polje. \\
    $a \leq b \iff a < b \vee a = b$ \\
    Opomba iz definicije direktno sledi, da so pozitivna števila vsa, za katera velja: $a > 0$.
\end{definicija}

\begin{trditev}
    Urejeno polje z relacijo $\leq$ je linearno urejeno.
\end{trditev}
\begin{dokaz}
    \begin{itemize}
        \item $a \leq a$ \\
        Sledi iz definicije.
        \item $a \leq b \wedge b \leq a \implies a = b$ \\
        Torej obstajata $a, b$, da bo $a < b$ in $b < a$.
        Iz tega sledi, da sta $a - b$ in $b - a$ obe pozitivni, kar je protislovje.
        \item $a \leq b \wedge b \leq c \implies a \leq c$ \\
        Glejmo primer, kjer so si $a, b, c$ različni, ker če ne je trivialno.
        $a < b \wedge b < c \implies b - a \in P \wedge c - b \in P \implies b - a + c - b \in P \implies c - a \in P \implies a < c$
        \item $a \leq b \vee b \leq a$ \\
        Če sta $a, b$ enaka je spet trivialno.
        Naj bosta potem različna.
        $a < b \vee b < a \implies b - a \in P \wedge a - b \in P$, kar je pa protislovje.
    \end{itemize}
\end{dokaz}

\begin{trditev}
    Za urejeno polje velja pravilo seštevanja in množenja. \\
    $a < b \wedge c < d \implies a + c < b + d$ \\
    $a < b \wedge c \in P \implies ac < bc$ \\
    $a < b \wedge c < d \wedge a, b, c, d \in P \implies ac < bd$
\end{trditev}
\begin{dokaz}
    Najprej seštevanje.
    $a < b \wedge c < d \implies b - a + d - c \in P \implies b + d - (a + c) \in P \implies a + c < b + d$ \\
    Še množenje.
    $a < b \wedge c \in P \implies c(b - a) \in P \implies cb - ca \in P \implies ac < bc$ \\
    $a < b \wedge c < d \wedge a, b, c, d \in P \implies ac < bc \wedge bc < bd \implies ac < bd$
\end{dokaz}

\begin{trditev}
    Kvadrat je vedno pozitiven. \\
    $a \neq 0 \implies a^2 \in P$
\end{trditev}
\begin{dokaz}
    Če je $a$ pozitiven, je trivialno
    Če je $-a$ pozitiven, je $(-a)(-a) = a^2$ pozitiven.
\end{dokaz}
