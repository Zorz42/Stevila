\begin{definicija}
    Naj bo $A$ množica in $\circ: A \times A \to A$ funkcija. \\
    Namesto $\circ((a, b))$ pišemo $a \circ b$. \\
    Če velja:
    \begin{itemize}
        \item $a \circ (b \circ c) = (a \circ b) \circ c$ (asociativnost)
        \item $\exists e \in A. a \circ e = e \circ a = a$ (obstoj enote)
    \end{itemize}
    potem je $(A, \circ)$ monoid.
\end{definicija}

\begin{definicija}
    Monoid $(A, \circ)$ je komutativen, če velja:
    \begin{itemize}
        \item $a \circ b = b \circ a$
    \end{itemize}
\end{definicija}

\begin{definicija}
    Naj bo $(A, \circ)$ monoid.
    Če dodatno velja:
    \begin{itemize}
        \item $\forall a \in A. \exists a^{-1} \in A. a \circ a^{-1} = a^{-1} \circ a = e$
    \end{itemize}
    potem je $(A, \circ)$ grupa.
\end{definicija}

\begin{definicija}
    Naj bo $(A, \circ)$ grupa.
    Če velja:
    \begin{itemize}
        \item $\forall a, b \in A. a \circ b = b \circ a$
    \end{itemize}
    potem je $(A, \circ)$ Abelova grupa.
\end{definicija}

\begin{trditev}
    Naj bo $(A, \circ)$ monoid.
    Enota je enolično določena.
\end{trditev}
\begin{dokaz}
    Predpostavimo, da obstajata dve enoti: $e$ in $e'$.
    Potem $e = e \circ e' = e'$.
\end{dokaz}

\begin{trditev}
    Naj bo $(G, \circ)$ grupa.
    Potem ima enačba $a \circ x = b$ enolično rešitev za vsak $a, b \in G$.
\end{trditev}
\begin{dokaz}
    \begin{align*}
        a \circ x = b \iff \\
        a^{-1} \circ (a \circ x) = a^{-1} \circ b \iff \\
        (a^{-1} \circ a) \circ x = a^{-1} \circ b \iff \\
        e \circ x = a^{-1} \circ b \iff \\
        x = a^{-1} \circ b
    \end{align*}
\end{dokaz}

\begin{trditev}
    Naj bo $(G, \circ)$ grupa.
    Potem velja: $a\circ b = a\circ c \implies b = c$
\end{trditev}
\begin{dokaz}
    \begin{align*}
        a\circ b = a\circ c \implies \\
        a^{-1} \circ (a \circ b) = a^{-1} \circ (a \circ c) \implies \\
        (a^{-1} \circ a) \circ b = (a^{-1} \circ a) \circ c \implies \\
        e \circ b = e \circ c \implies \\
        b = c
    \end{align*}
\end{dokaz}

\begin{trditev}
    Naj bo $(G, \circ)$ grupa, potem je inverz enolično določen za vsak $a \in G$ posebej.
\end{trditev}
\begin{dokaz}
    Predpostavimo, da obstajata dva inverza: $b$ in $b'$.
    Potem $b = b \circ e = b \circ (a \circ b') = (b \circ a) \circ b' = e \circ b' = b'$
\end{dokaz}

\begin{trditev}
    Naj bo $(G, \circ)$ grupa.
    Potem velja: $e^{-1} = e$.
\end{trditev}
\begin{dokaz}
    $e = e \circ e$, zato je $e$ inverz za $e$.
    Ker je inverz enolično določen, je $e^{-1} = e$.
\end{dokaz}

\begin{trditev}
    Naj bo $(G, \circ)$ grupa.
    Potem za vsak $a \in G$ velja: $(a^{-1})^{-1} = a$.
\end{trditev}
\begin{dokaz}
    \begin{align*}
        (a^{-1})^{-1} \circ a^{-1} = e = a^{-1} \circ a \implies \\
        (a^{-1})^{-1} \circ a^{-1} = a \circ a^{-1} \implies \\
        (a^{-1})^{-1} = a
    \end{align*}
\end{dokaz}

\begin{definicija}
    Naj bo $A$ množica in $\leq \subseteq A \times A$ relacija.
    Potem je $(A, \leq)$ delno urejena množica, če velja:
    \begin{itemize}
        \item $a \leq a$ (refleksivnost)
        \item $a \leq b \wedge b \leq a \implies a = b$ (antisimetričnost)
        \item $a \leq b \wedge b \leq c \implies a \leq c$ (tranzitivnost)
    \end{itemize}
    Če še dodatno velja:
    \begin{itemize}
        \item $a \leq b \vee b \leq a$ (sovisnost)
    \end{itemize}
    je $(A, \leq)$ linearno urejena množica.
\end{definicija}

\begin{definicija}
    $(A, +, \cdot)$ je polje, če sta $(A, +)$ in $(A, \cdot)$ Abelovi grupi, ter velja $a(b + c) = ab + ac$.
\end{definicija}

\begin{definicija}
    $(A, +, \cdot, P)$ je urejena polje, če je $(A, +, \cdot)$ polje, ter velja:
    \begin{itemize}
        \item ali je $a \in P$ ali pa $-a \in P$ za vse $a \neq 0$
        \item $0 \notin P$
        \item $a, b \in P \implies a + b \in P$
        \item $a, b \in P \implies ab \in P$
    \end{itemize}
\end{definicija}
