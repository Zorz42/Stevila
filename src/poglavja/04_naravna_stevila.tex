\begin{definicija}
    Naslednjiška funkcija: $s: a \longrightarrow a \cup \{a\}$

    Peanovi Aksiomi:
    \begin{itemize}
        \item $\emptyset = 0 \in \mathbb{N}$
        \item $a \in \mathbb{N} \implies s(a) \in \mathbb{N}$
        \item $(A \subseteq \mathbb{N} \wedge 0 \in A \wedge n \in A \implies s(n) \in A) \implies A = \mathbb{N}$
    \end{itemize}
\end{definicija}

\begin{trditev}
    Obstaja natanko ena množica, ki zadošča Peanovim Aksiomom.
\end{trditev}
\begin{dokaz}
    Obstoj:
    Po aksiomu neskončnosti, obstaja vsaj ena množica, ki zadošča prvima dvema aksiomoma.
    Označimo jo z $N$.
    Definirajmo $S = \{x$ zadosti prvima dvema aksiomoma $| x \subseteq N\}$
    $S$ je neprazna, saj $N \in S$, zato lahko rečemo $\mathbb{N} = \bigcap S$
    Predpostavimo, da zadnji aksiom ne drži, torej obstaja nek $A \subset \mathbb{N}$, ki izpolnjuje prve dva aksioma, ki ni enak $\mathbb{N}$.
    To je protislovje, saj $A \subset N$, zato bi $A$ moral biti v preseku, torej $N \subseteq A$, kar je pa protislovje, torej $\mathbb{N}$ res izpolnjuje zadnji aksiom.

    Enoličnost:
Predpostavimo, da obstajata dve množici, ki zadoščata Peanovim Aksiomom: $\mathbb{N}_1$ in $\mathbb{N}_2$.
V zadnji aksiom vstavimo $\mathbb{N}_1$ in $\mathbb{N}_2$ in dobimo $\mathbb{N}_1 = \mathbb{N}_2$.
\end{dokaz}

\begin{definicija}
    Množica naravnih števil je množica, ki zadošča Peanovim Aksiomom.
    Označimo jo z $\mathbb{N}$.
\end{definicija}

\begin{trditev}
    Naj bo $T(n)$ trditev za naravno število $n$.
    Potem sta naslednji trditvi ekvivalentni:
    \begin{itemize}
        \item $T(0)$ in $\forall n \in \mathbb{N}. T(n) \implies T(s(n))$
        \item $\forall n \in \mathbb{N}. T(n)$
    \end{itemize}
\end{trditev}
\begin{dokaz}
    $\implies$: \\
    Naj bo $A = \{n \in \mathbb{N} | T(n)\}$.
    Po tretjem aksiomu velja $A = \mathbb{N}$ in zato $\forall n \in \mathbb{N}. T(n)$.
    $\impliedby$: \\
    Očitno.
\end{dokaz}

\begin{trditev}
    $\forall x \in \mathbb{N}. x = 0 \vee \exists y \in \mathbb{N}. s(y) = x$
\end{trditev}
\begin{dokaz}
    Uporabimo indukcijo: Za $x = 0$ trditev drži.
    Predpostavimo, da trditev drži za $x$ očitno velja, za $s(x)$, saj je $x = y$.
\end{dokaz}

\begin{trditev}
    Preslikava $s$ je injektivna.
\end{trditev}
\begin{dokaz}
    \begin{align*}
        s(a) = s(b) \implies \\
        \{a,\{a\}\} = \{b,\{b\}\}
    \end{align*}
    1. $a = b \wedge \{a\} = \{b\}$ je vredu. \\
    2.
    \begin{align*}
        a = \{b\} \wedge b = \{a\} \implies \\
        b = \{\{b\}\}
    \end{align*}
    Treba je samo še dokazati, da nobeno naravno število nima natanko enega elementa.
    Za $0$ je očitno, saj $0 = \{\}$.
    Za $x \neq 0$ obstaja $y$, da bo $s(y) = x$, torej $x = \{y, \{y\}\}$.
    Predpostavimo, da ima $x$ en element, torej $y = \{y\}$, kar je pa protislovje.
\end{dokaz}

\begin{trditev}
    $\nexists a. s(a) = 0$
\end{trditev}
\begin{dokaz}
    $s(a) = 0 \implies \{a, \{a\}\} = \emptyset$ ampak $a \in \{a, \{a\}\} \wedge a \notin \emptyset$
\end{dokaz}

\begin{trditev}
    Naj bo $A \subseteq \mathbb{N}$ in $a \neq \emptyset$.
    Potem je $(\exists n \in \mathbb{N}. n \notin A \wedge s(n) \in A) \vee 0 \in A$.
\end{trditev}
\begin{dokaz}
    Predpostavimo, da za množico ne velja $(\exists n \in \mathbb{N}. n \notin A \wedge s(n) \in A) \vee 0 \in A$. \\
    Potem dobimo: $(n \notin A \implies s(a) \notin A) \wedge 0 \notin A$.
    Z indukcijo dokažemo, da nobeno naravno število ni v $A$.
    Za $0$ je očitno, za $s(n)$ pa sledi iz predpostavke.
    Iz tega sledi, da če množica ne zadostuje pogoju, je prazna.
\end{dokaz}

\begin{trditev}
    (Univerzalna lastnost naravnih števil) \\
    Naj bo $A$ neprazna množica, $f: A \to A$ preslikava in $a \in A$ nek fiksen element.
    Potem obstaja natanko ena preslikava $\sigma: \mathbb{N} \to A$, da bo $\sigma \circ s = f \circ \sigma$ in $\sigma(0) = a$.
\end{trditev}
\begin{dokaz}
    Treba bo skonstruirati $\sigma \subseteq \mathbb{N} \times A$, da bo veljalo:
    \begin{itemize}
        \item $\sigma$ je preslikava
        \item $(0, a) \in \sigma$
        \item $(n, m) \in \sigma \implies (s(n), f(m)) \in \sigma$
    \end{itemize}
    Naj bo $S \subseteq \mathcal{P}(\mathbb{N} \times A)$, ki vsebuje vse podmnožice, ki zadoščajo drugi in tretji točki.
    $S$ je neprazna, saj vsebuje vsaj $\mathbb{N} \times A$.
    Vzemimo $\sigma = \bigcap S$.
    Dokažimo, da je $\sigma$ preslikava: \\
    Najprej $\forall n \in \mathbb{N}. \exists b \in A. (n, b) \in \sigma$: \\
    Naj bo $T \subseteq \mathbb{N}$, kjer $\forall t \in T. \nexists b \in A. (t, b) \in \sigma$.
    Predpostavimo, da $T$ ni prazna, potem obstaja $t$, kjer $t = 0$ ali pa $l \notin T \wedge s(l) = t$. \\
    1. $t = 0$ pomeni, da je $(0, b) \notin \sigma$ za vse $b \in A$ kar je protislovje. \\
    2. $s(l) = t$ pomeni, da je $(s(l), f(m)) \notin \sigma$ za vse $m \in A$, kar je protislovje, saj obstaja $b \in A$, da je $(l, b) \in \sigma$. \\
    Dokažimo še enoličnost za vsak element v $A$ z indukcijo na $n$. \\
    1. $n = 0$.
    Denimo, da obstaja $b \in A$, da bo $(0, b) \in \sigma$ in $b \neq a$.
    Ampak potem $\sigma \setminus {(0, b)}$ zadošča drugi in tretji točki, kar je protislovje, saj je $\sigma$ presek vseh takih množic. \\
    2. $n = s(l)$ in za $l$ velja, da je enoličen.
    Mora veljati: $(l, m) \in \sigma \implies (n, f(m)) \in \sigma$.
    Če rečemo, da je $m$ tisti enoličen element (po predpostavki), ki zadošča $(l, m) \in \sigma$, potem dobimo $(n, f(m)) \in \sigma$.
    Denimo, da obstaja še en $m'$, ki zadošča $(l, m') \in \sigma$ in $m' \neq m$.
    Potem $\sigma \setminus \{(l, m')\}$ zadošča drugi in tretji točki, kar je protislovje. \\
    Zdaj vemo, da je $\sigma$ preslikava. \\
    Dokažimo še, da je $\sigma$ enolična.
    Naj bo $\tau$ še ena preslikava, ki zadošča pogojem.
    Potem velja: $\forall n \in \mathbb{N}. \sigma(n) = \tau(n)$.
    Dokažimo z indukcijo na $n$. \\
    1. $n = 0$.
    $\sigma(0) = a = \tau(0)$. \\
    2. $n \to s(n)$.
    $\sigma(s(n)) = f(\sigma(n)) = f(\tau(n)) = \tau(s(n))$. \\
    Torej je $\sigma = \tau$.
\end{dokaz}

\begin{definicija}
    Seštevanje na naravnih številih je preslikava $+: \mathbb{N} \times \mathbb{N} \to \mathbb{N}$, ki zadošča naslednjim aksiomom:
    \begin{itemize}
        \item $n + 0 = n$
        \item $n + s(m) = s(n + m)$
    \end{itemize}
\end{definicija}

\begin{trditev}
    Seštevanje na naravnih številih je dobro definirano.
\end{trditev}
\begin{dokaz}
    Fiksiramo $n$ in dokažemo, da je $m \to n + m$ enolično.
    V univerzalno lastnost naravnih števil vstavimo: $A = \mathbb{N}$, $f = s$, $a = n$.
    Potem obstaja natanko ena preslikava $\sigma: \mathbb{N} \to \mathbb{N}$, da bo $\sigma \circ s = s \circ \sigma$ in $\sigma(0) = n$,
    kar je ravno preslikava, ki jo iščemo.
\end{dokaz}

\begin{trditev}
    $(\mathbb{N}, +)$ tvori komutativni monoid.
\end{trditev}
\begin{dokaz}
    \begin{itemize}

    \item Asociativnost: $a + (b + c) = (a + b) + c$ \\
    Indukcija na $c$: \\
    1. $c = 0$: $a + (b + 0) = a + b = (a + b) + 0$ \\
    2. $c \to s(c)$: $a + (b + s(c)) = a + s(b + c) = s(a + (b + c)) = s((a + b) + c) = s(a + (b + c)) = (a + b) + s(c)$ \\

    \item Enota: $a + 0 = 0 + a = a$ \\
    $a + 0 = a$ sledi iz definicije. $0 + a = a$ bomo dokazali z indukcijo. \\
    1. $a = 0$ je očitno $0 + 0 = 0$ \\
    2. $a \to s(a)$. $0 + s(a) = s(0 + a) = s(a)$ \\

    \item Vmesna trditev: $s(a) + b = a + s(b)$ \\
    Indukcija na $b$: \\
    1. $b = 0$: $s(a) + 0 = s(a + 0) = a + s(0)$ \\
    2. $b \to s(b)$: $s(a) + s(b) = s(s(a) + b) = s(a + s(b)) = a + s(s(b))$ \\

    \item Komutativnost: $a + b = b + a$ indukcija na $a$. \\
    1. $a = 0$: smo že dokazali $0 + b = b + 0$. \\
    2. $a \to s(a)$: $s(a) + b = a + s(b) = s(a + b) = s(b + a) = b + s(a)$.
    \end{itemize}
\end{dokaz}

\begin{definicija}
    $1 = s(0)$ in sledi, da $s(a) = a + 1$
\end{definicija}

\begin{definicija}
    Množenje na naravnih številih definiramo kot:
    \begin{itemize}
        \item $a0 = 0$
        \item $a(b + 1) = ab + a$
    \end{itemize}
\end{definicija}

\begin{trditev}
    Množenje na naravnih številih je dobro definirano.
\end{trditev}
\begin{dokaz}
    Fiksiramo $a$ in dokažemo, da je operacija enolična.
    V univerzalno lastnost naravnih števil vstavimo: $A = \mathbb{N}$, $f(n) = n + a$, $a = 0$.
    Potem obstaja natanko ena preslikava $\sigma: \mathbb{N} \to \mathbb{N}$, da bo $\sigma \circ s = s \circ \sigma$ in $\sigma(0) = 0$,
    kar je ravno preslikava, ki jo iščemo.
\end{dokaz}

\begin{trditev}
    $0a = 0$
\end{trditev}
\begin{dokaz}
    Dokaz z indukcijo: \\
    1. $a = 0$: $0\cdot 0 = 0$. \\
    2. $a \to a + 1$: $0(a+1) = 0a + 0 = 0$.
\end{dokaz}

\begin{trditev}
    $(a + 1)b = ab + b$
\end{trditev}
\begin{dokaz}
    Dokaz z indukcijo: \\
    1. $b = 0$: $(a + 1)0 = 0 = a0 + 0$. \\
    2. $b \to b + 1$: $(a + 1)(b + 1) = (a + 1)b + a + 1 = ab + b + a + 1 = ab + a + b + 1 = a(b + 1) + b + 1$
\end{dokaz}

\begin{trditev}
    $a(b + c) = ab + ac$
\end{trditev}
\begin{dokaz}
    Dokaz in indukcijo na $a$: \\
    1. $a = 0$. $0(b + c) = 0 = 0b + 0c$. \\
    2. $a \to a + 1$. $(a + 1)(b + c) = a(b + c) + b + c = ab + b + ac + c = (a + 1)b + (a + 1)c$
\end{dokaz}

\begin{trditev}
    $(\mathbb{N}, \cdot)$ tvori komutativni monoid.
\end{trditev}
\begin{dokaz}
    \begin{itemize}

    \item Asociativnost: $a(bc) = (ab)c$ \\
    Indukcija na $c$: \\
    1. $c = 0$: $a(b0) = a0 = 0 = (ab)0$ \\
    2. $c \to c + 1$: $a(b(c + 1)) = a(bc + b) = abc + ab = ab(c + 1)$ \\

    \item Enota: $a1 = 1a = a$ \\
    $a1 = a$ bomo dokazali z indukcijo. \\
    1. $a = 0$ je iz trditve $0\cdot a = 0$ \\
    2. $a \to a + 1$. $(a + 1)1 = a1 + 1 = a + 1$ \\
    $1a = a$ bomo dokazali z indukcijo. \\
    1. $a = 0$ je iz definicije $a\cdot 0 = 0$ \\
    2. $a \to a + 1$. $1(a + 1) = 1a + 1 = a + 1$ \\

    \item Komutativnost: $ab = ba$ indukcija na $a$. \\
    1. $a = 0$: smo že dokazali $0b = 0 = b0$. \\
    2. $a \to a + 1$: $(a + 1)b = ab + b = ba + b = b(a + 1)$.
    \end{itemize}
\end{dokaz}

\begin{trditev}
    Pravilo krajšanja: $a + b = a + c \implies b = c$
\end{trditev}
\begin{dokaz}
    Indukcija na $a$: \\
    1. $a = 0$: trivialno velja. \\
    2. $a \to a + 1$: $(a + 1) + b = (a + 1) + c \implies s(a + b) = s(a + c) \implies a + b = a + c \implies b = c$
\end{dokaz}

\begin{trditev}
    $a + b = 0 \implies a = b = 0$
\end{trditev}
\begin{dokaz}
    BŠS predpostavimo, da $a \neq 0$ in potem $s(a') = a$, torej $a = a' + 1$. \\
    Dobimo: $a' + 1 + b = s(a' + b) = 0$, kar je pa protislovje.
\end{dokaz}

\begin{definicija}
    Definiramo relacijo $\leq \subset \mathbb{N} \times \mathbb{N}$ na naravnih številih: \\
    $a \leq b \iff \exists c \in \mathbb{N}. a + c = b$
\end{definicija}

\begin{trditev}
    $(\mathbb{N}, \leq)$ je linearno urejena množica.
\end{trditev}
\begin{dokaz}
    \begin{itemize}
        \item $a \leq a$ \\
        Očitno če v definicijo vstavimo $c = 0$.
        \item $a \leq b \wedge b \leq c \implies a \leq c$ \\
        $a + d = b$ in $b + d' = c$, zato $a + (d + d') = c$
        \item $a \leq b \wedge b \leq a \implies a = b$ \\
        $a + c = b$ in $b + c' = a$, torej $b + (c + c') = b \implies c + c' = 0 \implies c = c' = 0 \implies a = b$
        \item $a \leq b \vee b \leq a$ \\

    \end{itemize}
\end{dokaz}


