\begin{definicija}
    Naslednjiška funkcija: $s: a \longrightarrow \{a, \{a\}\}$

    Peanovi Aksiomi:
    \begin{itemize}
        \item $\emptyset = 0 \in \mathbb{N}$
        \item $a \in \mathbb{N} \implies s(a) \in \mathbb{N}$
        \item $(A \subseteq \mathbb{N} \wedge 0 \in A \wedge n \in A \implies s(n) \in A) \implies A = \mathbb{N}$
    \end{itemize}
\end{definicija}

\begin{trditev}
    Obstaja natanko ena množica, ki zadošča Peanovim Aksiomom.
\end{trditev}
\begin{dokaz}
    Obstoj:
    Po aksiomu neskončnosti, obstaja vsaj ena množica, ki zadošča prvima dvema aksiomoma.
    Označimo jo z $N$.
    Definirajmo $S = \{x$ zadosti prvima dvema aksiomoma $| x \subseteq N\}$
    $S$ je neprazna, saj $N \in S$, zato lahko rečemo $\mathbb{N} = \bigcap S$
    Predpostavimo, da zadnji aksiom ne drži, torej obstaja nek $A \subset \mathbb{N}$, ki izpolnjuje prve dva aksioma, ki ni enak $\mathbb{N}$.
    To je protislovje, saj $A \subset N$, zato bi $A$ moral biti v preseku, torej $N \subseteq A$, kar je pa protislovje, torej $\mathbb{N}$ res izpolnjuje zadnji aksiom.

    Enoličnost:
Predpostavimo, da obstajata dve množici, ki zadoščata Peanovim Aksiomom: $\mathbb{N}_1$ in $\mathbb{N}_2$.
V zadnji aksiom vstavimo $\mathbb{N}_1$ in $\mathbb{N}_2$ in dobimo $\mathbb{N}_1 = \mathbb{N}_2$.
\end{dokaz}

\begin{definicija}
    Množica naravnih števil je množica, ki zadošča Peanovim Aksiomom.
    Označimo jo z $\mathbb{N}$.
\end{definicija}

\begin{trditev}
    $\forall x \in \mathbb{N}. x = 0 \vee \exists y \in \mathbb{N}. s(y) = x$
\end{trditev}
\begin{dokaz}
    Uporabimo indukcijo: Za $x = 0$ trditev drži.
    Predpostavimo, da trditev drži za $x$ očitno velja, za $s(x)$, saj je $x = y$.
\end{dokaz}

\begin{trditev}
    Preslikava $s$ je injektivna.
\end{trditev}
\begin{dokaz}
    \begin{align*}
        s(a) = s(b) \implies \\
        \{a,\{a\}\} = \{b,\{b\}\}
    \end{align*}
    1. $a = b \wedge \{a\} = \{b\}$ je vredu. \\
    2.
    \begin{align*}
        a = \{b\} \wedge b = \{a\} \implies \\
        b = \{\{b\}\}
    \end{align*}
    Treba je samo še dokazati, da nobeno naravno število nima natanko enega elementa.
    Za $0$ je očitno, saj $0 = \{\}$.
    Za $x \neq 0$ obstaja $y$, da bo $s(y) = x$, torej $x = \{y, \{y\}\}$.
    Predpostavimo, da ima $x$ en element, torej $y = \{y\}$, kar je pa protislovje.
\end{dokaz}

\begin{trditev}
    $\nexists a. s(a) = 0$
\end{trditev}
\begin{dokaz}
    $s(a) = 0 \implies \{a, \{a\}\} = \emptyset$ ampak $a \in \{a, \{a\}\} \wedge a \notin \emptyset$
\end{dokaz}


